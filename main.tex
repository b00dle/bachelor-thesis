\documentclass[pdftex,12pt,a4paper]{report}

%------------------------------------------------
%	PACKAGES
%------------------------------------------------

\usepackage{webis}

\usepackage{ucs}

\usepackage[utf8x]{inputenc}

\usepackage{ngerman}

\usepackage[]{algpseudocode}

\usepackage{algorithm}

%------------------------------------------------
%	CONFIGURATION FOR COVER PAGE
%------------------------------------------------

\global\arbeit{Bachelorarbeit}

\global\titel{Mehrbenutzer Multi-Touch-Interaktion mit einem immersiven 3D Bildschirmtisch}

\global\subtitel{}

\global\bearbeiter{Sebastian Stickert}

\global\betreuer{Alexander Kulik, André Kunert}

\global\aufgabensteller{Prof. Dr. Bernd Fröhlich}

\global\abgabetermin{\today} % TODO insert abgabetermin

\global\ort{Weimar} % TODO insert abgabeort

\global\matrikelnummer{110882}

\global\geburtsdatum{21.04.1992}

\global\geburtsort{Rostock}

%------------------------------------------------
%	START DOCUMENT
%------------------------------------------------

\begin{document}


%------------------------------------------------
%	COVER PAGE
%------------------------------------------------

\deckblatt


%------------------------------------------------
%	DECLARATION FOR EXAMINATION OFFICE
%------------------------------------------------

\erklaerung


%------------------------------------------------
%	ABSTRACT
%------------------------------------------------

\begin{abstract}

Dieses Dokument dient als Muster für die Ausarbeitung einer Diplomarbeit
am Lehrstuhl für Webtechnologie und Informationssysteme. 
1 - 2 - 3 Ümläuttest.

\end{abstract}


%------------------------------------------------
%	TABLE OF CONTENTS
%------------------------------------------------

\tableofcontents


%------------------------------------------------
%	THESIS CHAPTERS
%------------------------------------------------

\chapter{Einführung}
\label{chp:einfuehrung}
Hier wird das Kapitel Einführung beschrieben. Das Kapitel besteht aus den Abschnitten \ref{sec:mehrbenutzer_vr_systeme}, \ref{sec:multi_touch_interaktion_mit_3d_szenen} und \ref{sec:ziel_der_arbeit}.

\section{Mehrbenutzer VR-Systeme}
\label{sec:mehrbenutzer_vr_systeme}

Hier steht Inhalt zu Mehrbenutzer VR-Systemen.


\section{Multi-Touch Interaktion mit dreidimensionalen Szenen}
\label{sec:multi_touch_interaktion_mit_3d_szenen}

Hier steht Inhalt zu Multi-Touch Interation mit dreidimensionalen Szenen.


\section{Ziel der Arbeit}
\label{sec:ziel_der_arbeit}

Hier steht Inhalt zum Ziel der Arbeit.

%------------------------------------------------

\chapter{Finger und Hand Tracking auf Grundlage von Maximally Stable Extremal Regions}
\label{chp:mser}
Im Rahmen dieser Arbeit wird ein Bildschirmtisch zur Projektion dreidimensionaler Inhalte verwendet, welcher Multi-Touch Eingaben erlaubt. Gegenüber handgehaltenen Geräten, wie Smartphones, oder Tablets, bietet dieser eine weit größere Interaktionsfläche. Diese begünstigt, wie in Abschnitt \ref{sec:mehrbenutzer_vr_systeme} beschrieben, die kollaborative Arbeit in einem Mehrbenutzerszenario. Die meisten multi-touch gesteuerten Eingabegeräte ordnen alle erkannten Fingerpositionen und deren Bewegung derselben Geste zu. Bei gleichzeitiger Bedienung durch mehrere Nutzer, können nach diesem Ansatz leicht Eingabekonflikte entstehen. Um dem Problem entgegenzuwirken ist ein System erforderlich, welches Aussagen über die hierarchische Zuordnung der Eingabepositionen treffen kann.
\\\\
Hierzu wird in dieser Arbeit die von Ewerling et al. entwickelte Finger und Hand Tracking Implementierung, auf Grundlage des Maximally Stable Extremal Regions (MSER) Algorithmus verwendet \cite{matas:2004, ewerling:2012}. Der theoretische Ansatz dieser Lösung wird in Abschnitt \ref{sec:maximally_stable_extremal_regions} vorgestellt. Abschnitt \ref{sec:technische_voraussetzungen} beleuchtet die technischen Voraussetzungen zur Umsetzung des Multi-Touch Tisches, gefolgt von einem Überblick über die softwareseitige Integration des  MSER Trackings in das genutzte Applikationsframework in Abschnitt \ref{sec:implementierung_mser}. Zuletzt werden die Vorteile und Limitierungen des Systems in Abschnitt \ref{sec:diskussion_mser} diskutiert.


\section{Maximally Stable Extremal Regions}
\label{sec:maximally_stable_extremal_regions}

Maximally Stable Extremal Regions ist ein Ansatz entwickelt von Matas et al. zur Auswertung von Bildreihen bei der photogrammetrischen Stereo-Rekonstruktion \cite{matas:2004}. Hierbei sollen vor allem eigenschaftsbezogene Korrespondenzen zwischen Stereobildpaaren mit unterschiedlichem Blickpunkt gefunden werden. Ewerling et al. nutzen diese Erkenntnisse zur Erkennung von Fingerspitzen auf dem Bildschirm \cite{ewerling:2012}. Es erfolgt dabei außerdem eine Zuordnung von Fingern zu Händen durch den \emph{MSER Component Tree}.  Dieses Verfahren ist motiviert durch die visuelle Wahrnehmung des Hand- und Armschattens auf den Bildern, welche durch Tracking mit diffuser Infrarotbeleuchtung erfasst werden (siehe Abbildung XX.).
\\\\
Extremal Regions (ER) werden als geschlossene Gruppe von Pixeln bezeichnet, welche entweder eine höhere oder eine niedrigere Farbintensität als umliegende Bildpunkte aufweisen. Die ER $R_i$ ist eine Maximally Stable Extremal Region, wenn $R_i$ eine anderen RE ($R_{i-1}$) umschließt und es eine umschließende ER ($R_{i+1}$) für $R_i$ gibt. Es muss folglich $R_{i-1} \subset R_i \subset R_{i+1}$ gelten. Außerdem muss das Stabilitätskriterium ($s_\Delta(i)$) für $i$ ein lokales Minimum erreichen. Folgende Formel beschreibt die Berechnung von $s_\Delta(i)$:
\\\\
$s_\Delta(i) = \frac{ |R_{i+\Delta}| - |R_{i-\Delta}| } { |R_i| }$
\\\\ 
,wobei $\Delta$ eine durch den Anwender definierte Konstante ist.
\\\\
Ewerling et al. nutzen einen erweiterten Ansatz zur Erkennung der MSER in linearer Zeit, nach dem Vorbild von Nistér et al. \cite{nister:2008}. Bei der direkten Verwendung der Methode von Nistér et al. ergibt sich eine hierarchische Struktur zur Beschreibung  aller gefundenen MSER, basierend auf dem Stabilitätskriterium. Genannte Struktur wird als \emph{Component Tree} bezeichnet. Die Konstruktion dieses Component Trees sei jedoch nach Ewerling et al. unvollständig im Hinblick auf für die Beschreibung der Finger-Hand-Relation wichtige ER. Der erweiterte Algorithmus wurde daher entwickelt, um alle verfügbaren ER zu analysieren.
\\\\
Anhand der ER im Component Tree erfolgt die Ableitung der Fingerpositionen und deren Handzuordnung. Hierfür definieren Ewerling et al. zwei grundlegende Regeln:

\begin{enumerate}
\item Die dunkelste und gleichzeitig größte gefundene ER ist der Wurzelknoten des Component Trees
\item Für alle Kind-Eltern Beziehungen in der Struktur gilt, dass die Kindknoten kleiner sind und eine höhere Farbintensität aufweisen
\end{enumerate}

Wie in Abbildung XX. zu erkennen, kreieren Objekte in naher Distanz zum Bildschirm eine hellere ER als weiter entfernte. Nach den gelisteten Kriterien können folglich nur Blattknoten des Baumes Fingerspitzen sein. Um eine robuste Analyse zu erreichen, werden vermeidliche Kandidaten in der untersten Ebene des Baumes auf typische visuelle Eigenschaften von Fingeraufsetzpunkten getestet. Hierzu zählen das Prüfen auf runde Form und eine definierte Durchschnittsgröße der erkannten Fingerregion. Die Zuordnung der gefundenen Fingerspitzen zu Händen erfolgt über ein hierarchisches Clustering der Knoten im Component Tree.

\section{Technische Voraussetzungen}
\label{sec:technische_voraussetzungen}

Für die technische Umsetzung des in Abschnitt \ref{sec:maximally_stable_extremal_regions} beschriebenen MSER Systems im Kontext der Multi-Touch Erkennung wird in unserem Labor ein rückseitiges, diffuses Infrarotbeleuchtungssetup genutzt. Hierbei wird eine diffuse Infrarot emittierende Lichtquelle unterhalb des Bildschirmtisches angebracht. Ein Spiegel lenkt das Licht senkrecht auf eine matte und lichtdurchlässige Projektionsfläche. Das von dieser Ebene reflektierte Licht wird durch den Spiegel gelenkt und von einer Infrarotkamera zur weiteren Verarbeitung aufgenommen. Abbildung XX visualisiert das beschriebene Setup.
\\\\
Die Genauigkeit des Systems kann durch infrarotes Umgebungslicht, wie beispielsweise Anteile der Sonnenstrahlung, oder andere im Arbeitsraum befindliche Trackingsysteme, gestört werden. Ewerling et al. schlagen daher eine hochfrequente Modulation der Beleuchtung nach Moeller und Kerne vor \cite{ewerling:2012, moeller:2012}.  Nicht-uniforme Lichtintensität entlang der Tischfläche kann ebenfalls zu Problemen bei der Bildanalyse führen. Diese können zwar durch Filterverfahren abgeschwächt, jedoch nicht vollständig beseitigt werden. Eine gleichmäßige Ausleuchtung der Projektionsebene ist demnach ausschlaggebend für eine akkurate Evaluation der Kamerabilder.


\section{Implementierung}
\label{sec:implementierung_mser}
Es existiert eine Anbindung des MSER Algorithmus zur Touch Erkennung, an das am Lehrstuhl entstandene Applikationsframework Avango\footnote{http://www.avango.org/}. Die analysierten Eingabedaten werden basierend auf dem TUIO Protokoll\footnote{http://www.tuio.org/} an die Anwendung gesendet.
\\\\
In Abschnitt \ref{subsec:tuio_touch_protokoll} wird eine Übersicht über die mit dem TUIO Protokoll bereitgestellten Datenstrukturen gegeben. Abschnitt \ref{subsec:umgang_mit_jittering} beschreibt das Auftreten von Jittering bei der Eingabeerkennung und wie mit diesem Problem umgegangen werden kann.


\subsection{TUIO Touch Protokoll}
\label{subsec:tuio_touch_protokoll}

TUIO ist ein von Kaltenbrunner et al. entwickeltes Protokoll, welches speziell für den Umgang mit berührbaren Tabletop Eingabeschnittstellen kreiert wurde \cite{kaltenbrunner:2005}. Nach der Auswertung der Eingabepositionen durch den MSER Algorithmus, füllt die Protokollkommunikation verschiedene Stations im Avango Daemon mit den ermittelten Daten. Es existieren zwei Stationstypen. Der Erste hält Informationen über einen jeweiligen Finger. Dies beinhaltet eine normalisierte \emph{Fingerposition}, die zugewiesene \emph{SessionID}, LIST ALL PARAMETERS. Ein weiterer Stationstyp ist für die Verwaltung handspezifischer Details verantwortlich. Hierzu gehören die \emph{SessionID} der Hand, eine Liste von \emph{FingerSessionIDs} der zugewiesenen Finger, LIST ALL PARAMETERS.
\\\\
Eine Abbildung dieser Struktur ist nach dem Feldcontainer Konzept, welches Avango bereitstellt, in der Applikation umgesetzt. Für jeden der Stationstypen existiert demnach eine Klasse, welche durch Update der Station aktualisierte Eingabewerte für die Anwendung verfügbar macht. Die nachfolgende Abbildung XX zeigt den Aufbau der Klassen \emph{TUIOFinger} und \emph{TUIOHand}.


\subsection{Umgang mit Jittering}
\label{subsec:umgang_mit_jittering}

Die Verfolgung der Bewegung von Eingabepunkten basiert auf der Auswertung von Bilddaten. Hierbei ist die Präzision der Positionsermittlung an die Genauigkeit des MSER Algorithmus, sowie die Auflösung der Bilder gebunden. Infolgedessen können, selbst bei Auflegen eines Fingers ohne Positionsverschiebung, variierende Eingabepositionen entstehen. In einer Visualisierung der Eingabeposition, äußert sich dieser Zusammenhang als Zittern (Jittering) der Darstellungsgeometrie.
\\\\
Jittering kann bei Anwendung der Eingabe auf die Manipulation einer virtuellen Szene ungewollte Auswirkungen hervorrufen. Bindet man eine Szene beispielsweise an die relative Positionsänderung einer durch Jittering beeinträchtigten Eingabe, so überträgt sich das Zittern auf die Szene, was als störend für die Wahrnehmung empfunden werden kann. Die Initiierung einer Bewegung sollte folglich weitestgehend dem Anwender überlassen sein.
\\\\ 
Um Jittering zu vermindern wird vor der Übermittlung von Daten durch das TUIO Protokoll ein Filter auf die Eingabepositionen angewandt. Hierzu wird der von Casiez et al. vorgestellte \emph{1\euro{} Filter} verwendet. Dieser Ansatz überzeugt vor allem durch seine niedrige Rechenintensität, sowie die geringe Latenzanfälligkeit bei schnellen Bewegungen.


\section{Vorteile und Limitierungen}
\label{sec:diskussion_mser}

Das MSER-System dient zur Erkennung von Relationalen Eingabedaten. Es können somit Aussagen über die hierarchische Zuordnung einzelner Fingerpositionen zu aufgelegten Händen gemacht werden. Dieser Zusammenhang ist essenziell für die im Rahmen dieser Arbeit entstandenen Techniken. 
\\\\ 
Im Umgang mit der Toucherkennung wurden im Verlauf der Programmierarbeiten jedoch immer wieder verschiedene Probleme deutlich. Demnach wird eine Finger-Hand Zuweisung zwar unterstützt, nicht konsequent Stabil gehalten. Die Variation der Höhe der Handfläche über dem Tisch kann beispielsweise zum Auflösen der Verbindung einzelner Teile der Wurzel ER einer Hand führen. In diesem Fall stellt sich die Eingabe einer Hand mit fünf aufgelegten Fingern, als mehrere Einzelne Hände mit einem Finger dar (siehe Abbildung XX). 
\\\\
Das gleichzeitige Aufsetzen zweier Hände eines Nutzers wird nur dann problemfrei erkannt, wenn zwei separate ER für die Arme entstehen. Beugt sich der Anwender über die Tischplatte um mit beiden Händen in weit entfernten Bereichen zu interagieren, kann leicht eine Verschmelzung der ER um beide Arme entstehen. Dies beeinträchtigt eine kollisionsfreie Handerkennung weiter. Dementgegen ist die Erkennung und Verfolgung einzelner Eingabepunkte relativ stabil. Lediglich schnelle Bewegungen über die Tischfläche sind hierbei problematisch und führen zum Verlust von Fingerpositionen.

%------------------------------------------------

\chapter{Anforderungsanalyse}
\label{chp:anforderungsanalyse}
Hier wird das Kapitel Anforderungsanalyse beschrieben. Das Kapitel besteht aus den Abschnitten \ref{sec:mehrbenutzer_touch_eingaben}, \ref{sec:tiefenwahrnehmung}, \ref{sec:interaktionsziele}, \ref{sec:kontrolle_der_freiheitsgrade} und \ref{sec:visuelle_rueckmeldung}.


\section{Mehrbenutzer Touch-Eingaben}
\label{sec:mehrbenutzer_touch_eingaben}

Anforderung: Fehleingaben durch gleichzeitiges Interagieren verschiedener Nutzer mit dem Bildschirmtisch sollten vermieden werden.
\\\\
Die meisten Interaktionswerkzeuge können von nicht mehr als einem Nutzer bedient werden. Selbst wenn die Auswirkung der Manipulation die ganze Gruppe betrifft, so obliegt der Umgang mit dem Werkzeug einer Person. Im Gegensatz dazu, steht der Projektionstisch allen Nutzern gleichermaßen zur Verfügung. Da eine Vielzahl verschiedener Fingerpositionen auf dem Touchtisch registriert werden kann, besteht folglich die Möglichkeit, dass der Tisch zur gleichen Zeit von mehreren Personen Eingaben erhält. Diese Eingabekombinationen können leicht zu Manipulationen der Applikation führen, welche vom Nutzer nicht beabsichtigt waren und somit vermieden werden sollten.
\\\\
Abbildung XX zeigt ein konkretes Anwendungsszenario. Hier versammeln sich eine bestimmte Anzahl Nutzer um gemeinsam Eigenschaften einer virtuellen Welt zu untersuchen. Während ein Nutzer beidhändig eine Interaktionsgeste ausführt, setzt ein anderer Nutzer seine Hand auf einen markanten Inhalt der Szene. Dabei berührt er ebenfalls die Tischfläche. In diesem Fall sollte sichergestellt sein, dass das System die Semantik der Touchpunkte erkennt und eine sinnvolle Entscheidung bezogen auf die abgeleitete Interaktion trifft. 


\section{Tiefenwahrnehmung durch Stereo- und Bewegungsparalaxe}
\label{sec:tiefenwahrnehmung}

Anforderung: Konflikten der Tiefenwahrnehmung durch Berührung negativ paralaxer Bildareale sollte entgegen gewirkt werden.
\\\\
Anforderung: Die Interaktion mit positiv und negativ paralaxen Objekten sollte nicht durch Bewegungsparalaxe erschwert werden. 


\section{Interaktionsziele}
\label{sec:interaktionsziele}

Anforderung: Manipulationen der Applikation, sollten basierend auf dem Interaktionsziel des Nutzers bestimmt werden.


\section{Explizite und implizite Kontrolle der involvierten Freiheitsgrade}
\label{sec:kontrolle_der_freiheitsgrade}

Anforderung: Das System sollte auch für unerfahrene Nutzer leicht benutzbar sein.
\\\\
Anforderung: Es sollte dem Nutzer möglich sein, alle für das jeweilige Interaktionsziel erforderlichen Freiheitsgrade getrennt voneinander zu bedienen.
\\\\
Anforderung: implizite kontrolle????



\section{Visuelle Rückmeldung}
\label{sec:visuelle_rueckmeldung}

Anforderung: Der Umgang mit der Schnittstelle sollte mit visuellem Output unterstützt werden

%------------------------------------------------

\chapter{3D Interaktion mit Multitoucheingaben}
\label{chp:interaktion_mit_multitoucheingaben}
Die Interaktion mit Multi-Touch Eingaben wurde in einigen Veröffentlichungen thematisiert. Hierbei haben sich unterschiedliche Strategien zur Kontrolle der Freiheitsgrade bei der Manipulation von dreidimensionalen Szenen bewährt. Die im Rahmen dieser Arbeit entstandenen Ansätze zur Umsetzung einer Multi-Touch basierten 3D Navigation stützen sich auf diese Erkenntnisse.
\\\\
Dieses Kapitel stellt die in diesem Kontext wichtigsten verwandten Arbeiten vor. Hierzu beschreibt Abschnitt \ref{sec:related_sticky_tools} ein System zur 6DOF Interaktion mit Multi-Touch Tischen. In Abschnitt \ref{sec:related_balloon_selection} wird ein Ansatz zur Selektion von virtuellen Objekten, durch Touch Eingaben erläutert. Abschnitt \ref{sec:related_two_axis_valuator} erklärt die Umsetzung einer expliziten Geste zur Steuerung der 3D Rotation. Abschließend erfolgt in Abschnitt \ref{sec:diskussion_interaktion} eine Diskussion genannter Techniken.


\section{Sticky Tools}
\label{sec:related_sticky_tools}

Hancock et al. stellen ein System zur Steuerung von sechs Freiheitsgraden der dreidimensionalen Objekttransformation durch Multi-Touch Eingaben vor, welches sie Sticky Tools nennen \cite{hancock:2009}. Sie stützen sich dabei auf bekannte Techniken welche als \emph{force-based} bezeichnet werden. Hierbei soll das Gefühl erzeugt werden, die virtuelle Geometrie wie in der realen Welt durch direktes Anfassen zu manipulieren. 
\\\\
Die x- und y-Translation, uniforme Skalierung, sowie z-Rotation mit zwei Fingern hat sich im Alltag, durch den Umgang mit zweidimensionalen Touch-Anwendungen, bewährt. Es wird hierbei ein Initialkontakt mit der Geometrie bestimmt. Während der Bewegung der Finger, bleibt dieser durch Anwendung der genannten Transformationen, erhalten. Dieses Ankleben der Objekte an die Berührungspunkte auf dem Bildschirm bezeichnen Hancock et al. als \emph{Sticky Fingers}-Technik \cite{hancock:2007,hancock:2009}.
\\\\
In dreidimensionalen Szenen wird die Translation um die z-Komponente erweitert. Hancock et al. nutzen für die Visualisierung der virtuellen Szene eine zweidimensionale Darstellung. Durch die perspektivische Verzerrung wächst der projizierte Abstand zweier Punkte auf einem Objekt, je mehr sich die Geometrie der Projektionsebene nähert. Sticky Tools simuliert den physikalischen Umgang mit nicht-elastischen Objekten. Aus diesem Grund wird auf eine Geste zur Skalierung verzichtet. Folglich wird die 2D Skalierungsgeste zur z-Translation eines Objekts genutzt.
\\\\
Neben der Rotation um die Hochachse ermöglicht die Manipulation in dreidimensionalen Szenen das Drehen von Objekten um Achsen auf der Bildfläche. Hancock et al. definieren letztere als \emph{flip}-Rotation. Um das Sticky Fingers Kriterium zu erhalten, wird diese Art der Rotation durch beidhändige Interaktion gesteuert. Hierbei werden durch das Aufsetzen zweier Finger einer Hand die direkten Kontaktpunkte auf der Geometrie festgelegt. Diese spannen einen Richtungsvektor auf, welcher die Rotationsachse definiert. Die Bewegung eines aufgesetzten Fingers der zweiten Hand im rechten Winkel zur Rotationsachse, legt den Grad der Flip-Rotation, sowie die Richtung der Drehung fest (siehe Abbildung \ref{fig:opposable_thumbs}).

\begin{figure}
	\begin{center}
		\includegraphics[width=12cm]{img/opposable_thumbs.pdf}
	\end{center}
	\caption{Durchführung der Opposable Thumbs Rotation im von Hancock et al. entwickelten System Sticky Tools \cite{hancock:2009}. Screenshots sind dem zugehörigen Video entnommen und durch eigene Visualisierungen erweitert worden \cite{hancock:2009:vid}.}
	\label{fig:opposable_thumbs}
\end{figure}


\section{Balloon Selection}
\label{sec:related_balloon_selection}

\begin{figure}
	\begin{center}
		\includegraphics[width=8cm]{img/baloon_concept.pdf}
	\end{center}
	\caption{Aufstiegsveränderung eines mit Helium gefüllten Ballons.}
	\label{fig:baloon_concept}
\end{figure}

Balloon Selection ist eine von Benko und Feiner \cite{benko:2007} entwickelte Technik zur Multi-Touch basierten Selektion von virtuellen Objekten in dreidimensionalen Szenen. Dieser Ansatz ist inspiriert vom Umgang mit einem Helium Ballon. Demnach wird das Ende der Leine eines solchen Ballons mit einem Finger auf den Boden gehalten. Ein zweiter Finger fixiert die Leine an einem anderen Punkt auf dieser Ebene. Die Länge des fixierten Stückes bestimmt die Aufstiegsdistanz des Ballons. Folglich kann diese Distanz, sowie die Position des Ballons über dem Boden, durch das Verschieben der Handpositionen variiert werden. Abbildung \ref{fig:baloon_concept} visualisiert diese Metapher.
\\\\
Bei der Anwendung dieser Idee für die Selektion, wird eine Kugel als Selektionsobjekt genutzt. Die abgeleitete Geste wird durch das Aufsetzen zweier Finger in unmittelbarer Nähe zueinander initiiert. Hierbei dient der zuerst aufgelegte Finger als Ankerpunkt (nach Benko und Feiner \emph{anchor}). Der zweite Finger (nach Benko und Feiner \emph{stretching finger}) wird zur Festlegung der Leinenlänge verwendet. Hierzu bewegt der Nutzer seine Hand vom Ankerpunkt weg. Die Interaktionsleine wächst bis diese Bewegung sich in entgegen gesetzte Richtung umkehrt. Ab diesem Punkt ist die Länge der Leine fest und die Kugel steigt um die Länge der Distanzverkürzung zwischen den beiden Fingern orthogonal zur Tischebene nach oben. Der Ankerpunkt bestimmt durch Positionsveränderung die x- und y- Position des Ballons über der Interaktionsfläche.


\section{Two Axis Valuator}
\label{sec:related_two_axis_valuator}

Rousset et al. beschreiben eine Erweiterung von Scheurich und Stuerzlingers \emph{Two Axis Valuator} (im Folgenden TAV genannt) \cite{scheurich:2013,rousset:2014}. Der von Elisabeth Rousset et al. entwickelte TAV+ ist ein Modus zur einhändigen 3D Rotation durch die Verwendung von zwei Fingern. 
\\\\
Für die Bestimmung der Rotation entlang einer beliebigen Achse auf der Bildebene, wird die Bewegung der Zentrumsposition zwischen den aufgesetzten Fingern verfolgt. Die Rotation wird hierbei um die Achse, welche rechtwinklig zur Bewegungsrichtung steht und ebenfalls in der Bildebene liegt, durchgeführt. Als Rotationszentrum dient der Schwerpunkt des zu manipulierenden Objekts. Drehrichtung und Rotationsgrad werden von der Distanz und Richtung der Translation des Fingerzentrums abgeleitet. Zusätzlich kann die Rotation um die Achse zwischen Fingerzentrum und Rotationszentrum kontrolliert werden. Dies ist durch Orientierungsänderung des Vektors zwischen den Fingern möglich. Der verwendete Winkel entspricht dem zwischen dem Vektor vor der Bewegung und demselben danach. Abbildung \ref{fig:tav_plus} illustriert dieses Verfahren.

\begin{figure}
	\begin{center}
		\includegraphics[width=10cm]{img/tav_plus.pdf}
	\end{center}
	\caption{Berechnung der dreidimensionalen Rotation TAV+ nach Rousset et al. \cite{rousset:2014}. Hierbei sind $A_i$ und $B_i$ die zwei zur Interaktion verwendeten Finger. $C_i$ ergibt sich als die Zenntrumsposition zwischen $A_i$ und $B_i$. $\alpha$ und $\beta$ sind die durch das Verfahren ermittelten Rotationswinkel. Diese Abbildung entstammt vollständig einer Veröffentlichung von Rousset et al \cite{rousset:2014}.}
	\label{fig:tav_plus}
\end{figure}


\section{Diskussion}
\label{sec:diskussion_interaktion}

Der direkte Umgang mit negativ-parallaxen Inhalten zeigt sich bei Touch-Interaktion mit stereoskopischen Visualisierungen als problematisch. Benko und Feiner kommen zu dem Schluss, dass Baloon Selection ein unter verschiedenen Umständen geeigneter Ansatz zur Selektion von über einer Interaktionsoberfläche liegenden virtuellen Objekten ist \cite{benko:2007}. Eine Umkehrung der Selektionsrichtung könnte das Verfahren auch für die 3D Multi-Touch Interaktion sinnvoll einsetzbar machen. Die Selektion von virtuellen Inhalten wird im Kontext dieser Arbeit nicht betrachtet. Jedoch ist eine explizite Höhenanpassung der Navigation, basierend auf den Erkenntnissen von Benko und Feiner, entstanden. Diese wird in Abschnitt \ref{sec:3d_translation} näher erläutert.
\\\\
Sticky Tools zeigt sich durch seine physikalisch motivierte Entwicklung als intuitiv und nutzerfreundlich für die 3D Objektmanipulation. Aus diesem Grund fließen einige der von Hancock et al. entwickelten Konzepte maßgeblich in die entstandene 3D Multi-Touch Navigation ein. Sticky Tools ersetzt die Funktionalität der herkömmlichen 2D Skalierungsgeste mit einer expliziten Höhenanpassung. Durch die perspektivische zweidimensionale Projektion geht hierbei visuell die direkte Verbindung zu den Kontaktpunkten auf der Geometrie nicht verloren. Stereoskopisches Rendering kann diese Illusion jedoch nicht gewährleisten. Zum Erhalt der Orthogonalität ist in diesem Anwendungsszenario eine uniforme Skalierung unabdingbar. 
\\\\
Der Umgang mit Objekten ist bei Sticky Tools auf geringe Entfernungen zur Bildebene begrenzt. Durch das Aufsetzen zweier Finger auf die Touch-Fläche werden Referenzpunkte auf der Geometrie festgelegt. Aus diesen Punkten wird eine Achse für die flip-Rotation abgeleitet. Weisen die Referenzpunkte einen Höhenunterschied auf, so wirkt sich dieser auf die hervorgerufene Drehung aus. Auf nahe Distanz zu manipulierenden Inhalten, ist die Höhendifferenz gut abschätzbar. Im Umgang mit komplexen Miniaturwelten und bei möglichen weiten Entfernungen zu Objekten, wird diese Abschätzung zunehmend schwerer. Es ist in diesem Kontext von einer Beeinträchtigung der Nutzbarkeit der flip-Rotation auszugehen.
\\\\
TAV+ wird von Rousset et al. als 3D Rotationstechnik, welche für unerfahrene Nutzer einfach bedienbar ist, bewertet. Im direkten Vergleich mit anderen 3D Rotationstechniken schneidet TAV+ gut ab. Hierbei wird der Ansatz im Szenario einfacher Observationsaufgaben einzelner Objekte mit festen Positionen untersucht. Die Wahl des Rotationszentrums im Schwerpunkt der Geometrie erscheint in diesem Zusammenhang als sinnvoll. Bei der Navigation wird der Blickpunkt auf die Szene und somit alle virtuellen Objekte fortlaufend geändert. In diesem Fall wäre die Bedienung der TAV+ Rotation mit festem Rotationsreferenzpunkt sicherlich wenig nützlich. Abschnitt \ref{sec:3d_rotation} beschreibt wie der von Rousset et al. entwickelte Ansatz abgeändert wurde, um eine effektive 3D Rotation der Navigation zu erreichen.


%------------------------------------------------

\chapter{Wahrnehmungskonflikte und Lösungsansätze}
\label{chp:wahrnehmungskonflikte_und_loesungsansaetze}
Hier wird das Kapitel Wahrnehmungskonflikte und Lösungsansätze beschrieben. Das Kapitel besteht aus den Abschnitten  \ref{sec:ansatz_1}, \ref{sec:ansatz_2} und \ref{sec:diskussion_wahrnehmungskonflikte}.


\section{Ansatz 1}
\label{sec:ansatz_1}

Hier steht Inhalt zu Ansatz 1.


\section{Ansatz 2}
\label{sec:ansatz_2}

Hier steht Inhalt zu Ansatz 2.


\section{Diskussion}
\label{sec:diskussion_wahrnehmungskonflikte}

Hier steht Inhalt zur Diskussion.

%------------------------------------------------

\chapter{Explizite Multi-Touch 3D Navigation}
\label{chp:explizite_interaktion}
In Kapitel \ref{chp:interaktion_mit_multitoucheingaben} werden einige bekannte Ansätze zur Interaktion durch Multi-Touch Eingaben vorgestellt. Diese werden im Rahmen dieser Arbeit als explizite 3D Multi-Touch Techniken bezeichnet. Wir definieren explizite Multi-Touch Navigation als Strategie, mit welcher verschiedene Freiheitsgrade der Manipulation durch Nutzereingaben direkt steuerbar sind.
\\\\
In diesem Kapitel wird beschrieben wie anhand dieser Techniken Gesten für die, im Rahmen dieser  Arbeit entstandene Applikation zur touch- basierten Navigation, abgeleitet wurden. Abschnitt \ref{sec:rst_im_bildraum} erläutert die Integration der Rotation, Translation und Skalierung im Bildraum, welche im Folgenden mit RTS abgekürzt wird. Abschnitt \ref{sec:3d_translation} zeigt wie alle Freiheitsgrade der dreidimensionalen Translation in einer Navigationsgeste bedient werden können. In Abschnitt \ref{sec:3d_rotation} wird ein Ansatz zur Steuerung aller Freiheitsgrade der dreidimensionalen Rotation vorgestellt. Abschließend werden in Abschnitt \ref{sec:vorteile_und_limitierungen_explizit} die vorgestellten Techniken gegenübergestellt und auf ihre Vorteile und Limitierungen untersucht.


\section{Rotation, Translation und Skalierung im Bildraum}
\label{sec:rst_im_bildraum}

Zur Umsetzung der RTS-Technik dienen die Erkenntnisse von Hancock et al. \cite{hancock:2007,hancock:2009}. Nach dieser Strategie soll zu jedem Zeitpunkt der Interaktion eine orthogonale Verbindung zwischen der Eingabeposition auf dem Bildschirm und der darunter liegenden Geometrie bestehen.

\begin{figure}
	\begin{center}
		\includegraphics[width=10cm]{img/screen-transformation.pdf}
	\end{center}
	\caption{Translation der virtuellen Bildfläche und die Auswirkung auf die Darstellung.}
	\label{fig:screen-transformation}
\end{figure}

\begin{figure}
	\begin{center}
		\includegraphics[width=10cm]{img/screen-scale.pdf}
	\end{center}
	\caption{Skalierung der virtuellen Bildfläche und die Auswirkung auf die Darstellung.}
	\label{fig:screen-scale}
\end{figure}

Wird zur Berechnung der Manipulation ein Kontaktpunkt auf dem Bildschirm verfolgt, können mit diesem Ansatz x- und y-Translation der Navigation gesteuert werden. Hierzu ergibt sich die Verschiebung des Viewing-Setups, aus der relativen Bewegung des Kontaktpunkts auf der Projektionsfläche. Die Abbildungen \ref{fig:screen-transformation} und \ref{fig:screen-scale} visualisieren die Auswirkung von Translation und Skalierung des virtuellen Bildschirms.
\\\\
Wir definieren $P_1(t) = (x_1, y_1)$ als die Position des Kontaktpunkts $P_1$ auf der Bildebene, zu einer gegebenen Zeit $t$. $P_1(t‘) = (x_1‘, y_1‘)$ sei die Position von $P_1$ zu einer späteren Zeit $t‘$. Die relative Translation $T_r = (x_r, y_r)$ berechnet sich durch $T_r = P_1(t) – P_1(t‘)$. Um diese Bewegung auszugleichen muss die Navigation um $T = (x_t, y_t) = (-x_r, -y_r)$ verschoben werden.
\\\\
Mit der Verwendung eines zweiten Kontaktpunkts $P_2$ auf dem Bildschirm, wird diese Transformation um die Rotation entlang der Bildschirmnormalen, sowie die symmetrische Skalierung erweitert. Als Referenzpunkt hierfür dient $P_1$.
\\\\
Sei $P_2(t) = (x_2, y_2)$ die Position des Kontaktpunkts $P_2$ auf der Bildebene, zu einer gegebenen Zeit $t$ und $P_2(t‘) = (x_2‘, y_2‘)$ die Position von $P_1$ zu einer späteren Zeit $t‘$, so ergibt sich $V = P_2(t) – P_1(t)$, als Richtungsvektor zwischen den Kontaktpunkten vor der Bewegung. $V‘ = P_2(t‘) – P_1(t‘)$ ist folglich der Richtungsvektor nach der Bewegung. Verändert sich die Distanz der Kontaktpunkte auf der Tischfläche, so muss sich die Distanz der Angriffspunkte auf der Geometrie in gleichem Maße ändern. Die Größenrelation der dargestellten Geometrie kann durch Skalierung der virtuellen Bildebene beeinflusst werden. Der Skalierungsfaktor $S$, bestimmt sich nach diesem Zusammenhang aus dem Verhältnis der Distanzen zwischen den Kontaktpunkten zu den Zeiten $t$ und $t‘$. Folglich gilt $S = |V| / |V‘|$. 
\\\\
Für die Rotation dienen ebenfalls die Vektoren $V$ und $V‘$ zur Berechnung. Als Achse dient die Bildschirmnormale $N$. Diese kann leicht durch $N = ||V|| \times ||V‘||$ bestimmt werden. Der Winkel $\alpha$ für die Transformation ist durch $\alpha = arccos(||V|| * ||V‘||)$ gegeben.


\section{3D Translation}
\label{sec:3d_translation}

Die 3D Translation ist ein von den Erkenntnissen der Balloon Selection nach Benko und Feiner \cite{benko:2007} abgeleiteter Ansatz. Durch \linebreak Berühren  der Tischfläche mit einer Hand wird eine direkte Verbindung mit der darunter liegenden Geometrie hergestellt. Diese zur Tischfläche orthogonale Beziehung soll zu jedem Zeitpunkt der Navigation in diesem Modus gewahrt bleiben. Es leitet sich demnach für die Interaktion mit nur einem Eingabepunkt eine zweidimensionale Translation, nach den in Abschnitt \ref{sec:rst_im_bildraum} beschriebenen Zusammenhängen, ab.
\\\\
Durch die Verfolgung der Bewegung zweier Eingabepositionen, kann eine dreidimensionale Translation abgebildet werden. Hierzu dient einer der verfolgten Kontakte auf dem Tisch als Primäreingabe. Die Bewegung dieser Position bestimmt weiterhin die x- und y-Translation im Bildraum. Der zweite Kontakt wird im Folgenden als Sekundäreingabe bezeichnet. Die Einordnung in Primär- und Sekundäreingabe kann durch Auswertung der Startzeit der jeweiligen Eingabe festgelegt werden. Durch Distanzveränderung der Sekundäreingabe zur Primäreingabe kann die z-Verschiebung gesteuert werden. Zur Berechnung dieser bieten sich zwei verschiedene Verfahren an.
\\\\
Wir definieren $P_1(t) = (x_1, y_1)$ und $P_2(t) = (x_2, y_2)$ als die Positionen der Primäreingabe $P_1$ und der Sekundäreingabe $P_2$ auf der Projektionsfläche, zu einer gegebenen Zeit $t$. $P_1(t‘) = (x_1‘, y_1‘)$ und $P_2(t‘) = (x_2‘, y_2‘)$  seien die Position von $P_1$ und $P_2$ zu einer späteren Zeit $t‘$. Weiterhin werden die Richtungsvektoren $V = P_2(t) – P_1(t)$ und $V‘ = P_2(t‘) – P_1(t‘)$ bestimmt. $z_t$ ist die zu ermittelnde z-Translation.
\\\\
Das erste Verfahren nutzt die Differenz zwischen $|V|$ und $|V‘|$ zur Bestimmung der z-Translation. Bei Verlängerung der Distanz zwischen den Eingabepunkten, soll der Abstand zur darunter liegenden Geometrie in gleichem Maße abnehmen. Bei Verkürzung der Strecke zwischen $P_1$ und $P_2$ wächst die Distanz zur Geometrie um den Betrag der Differenz. Für ein Viewing Setup, mit Blickrichtung entlang der negativen z-Achse, ergibt sich:
\\\\
$z_t = |V| – |V‘|$
\\\\
Der entstehende Effekt lässt sich mit der Anwendung eines Flaschenzugs vergleichen werden. Im zweiten Verfahren wird das Verhältnis zwischen $|V|$ und $|V‘|$ zur Bestimmung der z-Translation genutzt. Verdoppelt sich beispielsweise die Strecke zwischen $P_1$ und $P_2$, so halbiert sich die Distanz zwischen $P_1$ und dem Geometrieauftreffpunkt ($P_g$). $G$ ergibt sich als Richtungsvektor zwischen $P_1$ und $P_g$
\\\\
$z_t = |G| - |G| * \frac{|V|}{|V'|}$
\\\\
Für eine sinnvolle Einbindung beider Ansätze in die Applikation wird die skalierungs-basierte Berechnung von $z_t$ nur angewandt wenn \linebreak $|G| - |G| * \frac{|V|}{|V'|} > |V| – |V‘|$ gilt. 
\\\\
Abbildung \ref{fig:baloon_interaction} visualisiert den Umgang mit diesem Interaktionsmodus. Hierbei wird dem Nutzer eine Visualisierung des manipulierten Auftreffpunktes auf der Geometrie dargestellt, sowie die Verbindungslinie zur darüberliegenden Handposition. Die Gerade zwischen den Eingabepositionen wird ebenfalls visuell referenziert.

\begin{figure}
	\begin{center}
		\includegraphics[width=10cm]{img/baloon_interaction.pdf}
	\end{center}
	\caption{Interaktion im 3D Translationsmodus auf Grundlage der Erkenntnisse von Benko und Feiner \cite{benko:2007}. Die roten Pfeile sowie deren Beschriftung sind nicht Teil der Interaktionsvisualisierung}
	\label{fig:baloon_interaction}
\end{figure}

\section{3D Rotation}
\label{sec:3d_rotation}

Ziel dieser Technik ist die Kontrolle aller Freiheitsgrade, welche für die Rotation im dreidimensionalen Raum benötigt werden. Hierzu wurde der in Abschnitt \ref{sec:related_two_axis_valuator} vorgestellte Ansatz von Rousset et al. \cite{rousset:2014} implementiert. 
\\\\
Für die Bestimmung der Manipulation wird die Hand eines Nutzers mit genau zwei aufgesetzten Fingern verfolgt. Initiiert der Nutzer die Geste durch Aufsetzen der Hand, so wird ein direkter Angriffspunkt auf der Geometrie mit orthogonaler Verbindungsgeraden zur Bildschirmfläche berechnet. Bis zur Beendung der Geste durch Anheben der Hand erfolgt eine Rotation um diesen Punkt.
\\\\
Gegeben sind $P_h(t)$ und $P_h(t‘)$ als Positionen des Mittelpunkts der Geraden zwischen den zwei aufgesetzten Fingern zu den Zeiten $t$ und $t‘$. $P_g$ sei der beschriebene Referenzpunkt auf der Geometrie. Zuletzt sind $V_f(t)$ und $V_f(t‘)$, als zeitabhängigen Richtungsvektoren zwischen den Fingern, gegeben. Es werden für die Interaktion zwei Rotationen getrennt voneinander berechnet.

\begin{figure}
	\begin{center}
		\includegraphics[width=12cm]{img/3d_rotation.pdf}
	\end{center}
	\caption{Interaktion mit dem 3D Rotationsmodus nach Rousset et al. \cite{rousset:2014}.}
	\label{fig:3d_rotation}
\end{figure}

Die Richtungsvektoren $V_{hg}(t)$ und $V_{hg}(t‘)$ ergeben sich aus der Verbindung zwischen $P_h(t)$ mit $P_g$ und $P_h(t‘)$ mit $P_g$. Die Normale auf die von den Vektoren aufgespannte Ebene bildet die Achse der ersten Rotation. Der Winkel leitet sich aus dem Winkel zwischen $V_{hg}(t)$ und $V_{hg}(t‘)$ ab. $V_{hg}(t‘)$ ist außerdem die Achse für eine zweite Rotation, deren Maß durch den Winkel zwischen $V_f(t)$ und $V_f(t‘)$ bestimmt wird. Abbildung \ref{fig:3d_rotation} stellt das Verfahren an einem Beispiel dar.


\section{Vorteile und Limitierungen}
\label{sec:vorteile_und_limitierungen_explizit}

Durch die Verwendung der RTS-Technik (siehe Abschnitt \ref{sec:rst_im_bildraum}) sind eine Vielzahl verschiedener Transformationen im dreidimensionalen Raum gleichzeitig und auch getrennt voneinander zu bedienen. Der anhaltende und direkte Kontakt mit der Geometrie vermittelt dem Nutzer das Gefühl die Szene zu greifen, was zu einer intuitiven Einarbeitung in den Umgang mit dem System führt. Die Technik ist jedoch begrenzt auf dieselben Freiheitsgrade, welche auch im zweidimensionalen Raum verwendet werden. Sie bietet daher nicht die Möglichkeit, die Navigation in jeden durch die drei Dimensionen gegebenen Zustand zu bewegen. Dieser Nachteil spiegelt sich bei den übrigen in diesem Kapitel vorgestellten Techniken noch stärker wieder. Demzufolge ist durch 3D Translation und 3D Rotation jeweils nur eine Form der Transformation steuerbar. Im Kontext ihrer Anwendung zeigen sich beide Techniken ebenfalls als nutzerfreundliche Strategien zur Manipulationen der von ihnen bestimmten Freiheitsgrade.
\\\\
Die Flaschenzugstrategie bei der 3D Translation weist durch ihr direktes Mapping eine hohe Präzision und ein leicht Verständliches Interaktionskonzept auf. Durch die Maße des Tisches und die Reichweite des menschlichen Armes ist der Bewegungsrahmen für die Interaktion eingeschränkt. Ein weit entferntes Objekt in die Nähe der Projektionsebene zu bringen, erfordert somit das wiederholte Anheben und erneute Aufsetzen der Hand. Aus diesem Grund wirkt die Flaschenzugstrategie, bei der Arbeit mit weit entfernten Objekten, ungeeignet. Die Translationsberechnung durch das Verhältnis von Eingabepunkt- und Geometrieabstand kann hingegen effektiv für grobe Interaktionen mit weit entfernten Objekten genutzt werden. Kleine Bewegungen auf der Tischfläche führen zu einer zunehmenden Auswirkung, je weiter das berührte Objekt vom Tisch entfernt liegt. Dementgegen ist die Auswirkung von weitreichenden Bewegungen mit Geometrieelementen in Tischebene gering. Somit entstehen leicht Missverständnisse bei der Nutzung mit bildschirmnaher Geometrie.
\\\\
Die vorgestellte 3D Rotationstechnik kann den Erhalt der Orthogonalität, zwischen dem Geometrie-Eingabe-Vektor und der Bildfläche, nicht gewährleisten. Das hebt die Metapher der direkten Berührung auf, welche sich als nutzerfreundlich erwiesen hat. Wie in Kapitel \ref{chp:wahrnehmungskonflikte_und_loesungsansaetze} beschrieben ist 3D Multi-Touch Interaktion effektiv für Objekte in Null-Parallaxe verwendbar. Bei der 3D Rotation können geringe Eingabeänderungen zu starken Anpassungen des Rotationswinkels führen, wenn die Distanz zwischen Eingabeposition und Rotationsreferenzpunkt gering ist. Befindet sich der Referenzpunkt auf der Tischfläche ist gar keine Berechnung der Rotation mehr möglich. Des Weiteren wird durch den Übergang zwischen Positiv- und Negativ-Parallaxe die Steuerung der Rotation invertiert.


%------------------------------------------------

\chapter{Levelling: Implizite Multi-Touch 3D Navigation}
\label{chp:implizite_navigation}
Hier wird das Kapitel Levelling: Implizite Multi-Touch 3D Navigation beschrieben. Das Kapitel besteht aus den Abschnitten \ref{sec:depth_levelling}, \ref{sec:rotation_levelling} und \ref{sec:vorteile_und_limitierungen_implizit}.


\section{Depth-Levelling}
\label{sec:depth_levelling}

Hier steht Inhalt zu Depth-Levelling.


\section{Rotation-Levelling}
\label{sec:rotation_levelling}

Hier steht Inhalt zu Rotation-Levelling.


\section{Vorteile und Limitierungen}
\label{sec:vorteile_und_limitierungen_implizit}

Hier steht Inhalt zu Vorteilen und Limitierungen.

%------------------------------------------------

\chapter{Wechsel zwischen Interaktionstechniken}
\label{chp:wechsel_zwischen_interaktionstechniken}
Wie in den Kapiteln \ref{chp:explizite_interaktion} und \ref{chp:implizite_navigation} erläutert wird, können verschiedene implizite und explizite Navigationstechniken verwendet werden um alle Freiheitsgrade der Manipulation eines Viewing Setups in einer virtuellen Szene zu steuern. Durch die Vielzahl dieser unterschiedlichen Formen der Interaktion, ist es wichtig geeignete Kriterien zu definieren, nach denen zwischen den definierten Techniken gewechselt werden kann. 
\\\\
In Abschnitt \ref{sec:evaluierung_der_bildschirmkontakte} wird eine implizite Herangehensweise an diese Problematik vorgestellt. Diese bestimmt durch Evaluierung der vom Nutzer platzierten Kontaktpunkte auf dem Bildschirmtisch, welche der Interaktionstechniken für die Touch Navigation genutzt wird. Abschnitt \ref{sec:interaktionsterritorien} zeigt wie durch den Anwender Interaktionsterritorien zur Navigation in einem bestimmten Modus festgelegt werden können. Zuletzt werden in Abschnitt \ref{sec:diskussion_wechsel} beide Ansätze verglichen und auf ihre Stärken und Schwächen untersucht.



\section{Evaluierung der Bildschirmkontakte}
\label{sec:evaluierung_der_bildschirmkontakte}

Die in Kapitel \ref{chp:mser} beschriebene zugrunde liegende Touch-Eingabeerkennung gibt Aufschluss über vom Nutzer platzierte Bildschirmkontaktpunkte. Desweiteren ermöglicht Sie die Zuordnung dieser Fingerpositionen zu einzelnen Händen. Obwohl hierbei eine genaue Hand-Nutzer Zuweisung nicht möglich ist, kann die Evaluierung der gegebenen Eingabeinformation bereits effektiv als Kriterium zur Festlegung des Navigationszustands genutzt werden.
\\\\
In kollaborativen Anwendungsszenarien, ist gemeinsame Analyse der gebotenen virtuellen Inhalte stark an das Zeigen auf Bereiche der Visualisierung gebunden. Wie in der Anforderungsanalyse (siehe Kapitel \ref{chp:anforderungsanalyse}) vorgestellt, verlieren Nutzer durch Stereoparalaxe leicht die visuelle Wahrnehmung des physischen Bildschirms. In Folge dessen, kann es passieren das Nutzer versehendlich mit einem Finger die Bildschirmoberfläche berühren und damit ungewollt Manipulationen an der Applikation vornehmen. Um diesen Eingabekonflikt zu verhindern, wird die Eingabe einer Hand nur für die Interaktion angewandt, falls der Anwender mehr als einen Finger auf die Bildschirmoberfläche legt. 
\\\\
Da für die Berechnung der Manipulationsparameter nur ein Punkt der jeweiligen Hand als Eingabe nötig ist, wird eine Handzentrumsposition ermittelt. Diese wird aus den Positionen aller enthaltenen Finger berechnet. Sie kann wahlweise durch den gemittelten Positionswert aller Kontaktpunkte, oder den Mittelpunkt des fingerumschließenden Rechtecks, definiert sein. Für die Berechnung der Manipulationsparameter wird die relative Translation der gesamten Hand des Nutzers über den Bildschirm auf die jeweiligen Interaktionsmodi übertragen. Hierbei ist zu beachten, dass die berechnete Handzentrumsposition kein fester Punkt auf der Hand des Nutzers ist. Folglich führt das Aufsetzen oder Entfernen von Fingern an der interagierenden Hand ebenfalls zur Verschiebung des Handzentrums. Die Auswirkung dieser Verschiebung auf die Manipulation sollte verhindert werden, da sie zu sprunghaften Veränderungen der Applikation führt. Abbildung XX verdeutlicht diesen Zusammenhang.
\\\\
Vergleicht man alle in den Kapiteln \ref{chp:explizite_interaktion} und \ref{chp:implizite_navigation} vorgestellten Interaktionsmodi hinsichtlich ihrer benötigten Kontaktpunkte, so wird deutlich, dass keine der Techniken mehr als zwei  Inputpositionen über die Zeit verfolgt. Eingabekonflikte entstehen häufig, wenn mehrere Anwender gleichzeitig mit dem Bildschirmtisch interagieren wollen. Dieses Problem kann dadurch begrenzt werden, dass nicht mehr als zwei Hände zur Interaktion zugelassen werden. Demzufolge werden genau die beiden Hände zur Eingabe genutzt, welche den Projektionstisch früher berühren. Abbildung XX illustriert die Auswirkung dieser Technik auf ein beispielhaftes Szenario. 
\\\\
(PROBLEM FÜR EINE HAND UND EINTREFFENDEN 2. Input in diskussion beleuchten)
\\\\
Die beschriebenen Navigationstechniken lassen sich ebenfalls durch Auswertung der Kontaktpunkte modular auswählen. Das Aufsetzen eines einzelnen Fingers wird als Zeigegeste interpretiert und hat somit keine Auswirkung auf die Navigation. 2D Translation ist einhändig bedienbar. Gleiches gilt für die in Abschnitt \ref{sec:3d_rotation} beschriebene 3D Rotation falls die interagierende Hand mehr als einen Finger beinhaltet. Hierbei werden die Handzentrumsposition und der zugehörige Richtungsvektor zum Rotationsreferenzpunkt für die x- und y-Rotation verwendet. Während ein Vektor zwischen zwei Fingern der interagierenden Hand die explizite z-Rotation bestimmt. Letzterer Vektor ließe sich  leicht von 2 beliebigen Fingern der Hand berechnen. Zugunsten der Verständlichkeit der Interaktion und um eine klare Definition für die Wahl des Modus zu gewährleisten, wird der 3D Rotationsmodus ausschließlich bei der Eingabe mit zwei Fingern, bei genau einer Aufgelegten Hand gesteuert. Somit bleiben Dreifinger-, Vierfinger- und Fünffinger-Einhand-Interaktion für die Kontrolle der 2D Translation. Da für die in Kapitel \ref{chp:implizite_navigation} eingeführte implizite Levelling Technik die orthogonale geometrische Lage zwischen Bildschirm- und Geometrieauftreffpunkt gewährleistet werden soll, wird Levelling der expliziten Rotation, Translation und Skalierung im Bildraum beigefügt. Für diesen Modus werden zur Berechnung aller Manipulationsparameter zwei Kontaktpunkte benötigt. Selbiges gilt für die in Abschnitt \ref{sec:3d_translation} vorgestellte 3D Translation. Da letztere Interaktionstechnik jedoch keine Auswirkungen auf Skalierung und Rotation hat, wird ein allgemeiner Translationsmodus definiert. Dieser schließt die Verwendung von einhändiger 2D Translation ein. Setzt ein Nutzer nach einem bestimmten Zeitintervall der Einhand-Interaktion eine zweite Hand auf, wird die  3D-Translation angewandt. Demzufolge wird der Navigationsmodus für Rotation, Translation und Skalierung im Bildraum, zuzüglich Levelling nur betreten, wenn zwei Hände mit zeitlicher Differenz kleiner des bestimmten Zeitintervalls aufgesetzt werden. Ein Zustandsdiagramm (siehe Abbildung XX) verdeutlicht den Umgang mit den Navigationsmodi durch Evaluierung der Bildschirmkontakte.


\section{Interaktionsterritorien}
\label{sec:interaktionsterritorien}

Hier steht Inhalt zu Interaktionsterritorien.


\section{Diskussion}
\label{sec:diskussion_wechsel}

%------------------------------------------------

\chapter{Abbildung von Touch Eingaben in der Applikationsstruktur}
\label{chp:applikationsstruktur}
Hier wird das Kapitel Abbildung von Touch Eingaben in der Applikationsstruktur beschrieben. Das Kapitel besteht aus den Abschnitten \ref{sec:multitouch_input_pipeline}, \ref{sec:representation_im_szenegraph}, \ref{sec:touchwerkzeuge_und_applikationsmanipulatoren} und \ref{sec:diskussion_applikationsstruktur}.


\section{Multitouch-Input Pipeline}
\label{sec:multitouch_input_pipeline}

Hier steht Inhalt zur Multitouch-Input Pipeline.


\section{Representation im Szenegraph}
\label{sec:representation_im_szenegraph}

Hier steht Inhalt zur Representation im Szenegraph.


\section{Touchwerkzeuge und Applikationsmanipulatoren}
\label{sec:touchwerkzeuge_und_applikationsmanipulatoren}

Hier steht Inhalt zu Touchwerkzeugen und Applikationsmanipulatoren.


\section{Diskussion}
\label{sec:diskussion_applikationsstruktur}

Hier steht Inhalt zur Diskussion.

%------------------------------------------------

\chapter{Virtuelles Freischneiden der Hand während Berühungseingaben}
\label{chp:freischneiden}
Hier wird das Kapitel Virtuelles Freischneiden der Hand während Berühungseingaben beschrieben. Das Kapitel besteht aus den Abschnitten \ref{sec:ansatz}, \ref{sec:implementierung_freischneiden} und \ref{sec:vorteile_und_limitierungen_freischneiden}.


\section{Ansatz}
\label{sec:ansatz}

Hier steht Inhalt zum Ansatz.


\section{Implementierung}
\label{sec:implementierung_freischneiden}

Hier steht Inhalt zur Implementierung.


\section{Vorteile und Limitierungen}
\label{sec:vorteile_und_limitierungen_freischneiden}

Hier steht Inhalt zu Vorteilen und Limitierungen.

%------------------------------------------------

\chapter{Diskussion}
\label{chp:diskussion}
In diesem Kapitel soll eine abschließende Diskussion der im Rahmen dieser Arbeit entstandenen Konzepte zur multi-touch-basierten Interaktion mit stereoskopischen Bildinhalten erfolgen. Hierzu soll die Bewertung dieser Ansätze in Abschnitt \ref{sec:anforderungsevaluierung} anhand der in Kapitel \ref{chp:anforderungsanalyse} definierten Anforderungen bemessen werden. Abschließend werden in Abschnitt \ref{sec:hypothesen} einige Hypothesen definiert die aus Beobachtungen während dem Arbeitsprozess resultieren.


\section{Anforderungsevaluierung}
\label{sec:anforderungsevaluierung}

Der Grundstein für die Entwicklung der Touch Navigation ist durch das \emph{MSER Hand und Finger Tracking} gegeben (siehe Kapitel \ref{chp:mser}). Dieses System bietet eine Reihe von Informationen über die Struktur der Eingabe auf dem Bildschirmtisch. Demnach konnte eine Finger-Hand Zuordnung sinnvoll in der Applikationsstruktur abgebildet werden (siehe Kapitel \ref{chp:applikationsstruktur}). Durch Einführen eines Zusatzkriteriums beim Umgang mit Interaktionstechniken, wurde die maximale Anzahl an für die Manipulation zugelassenen Hände auf zwei begrenzt. Eine Hand-Nutzer Zuordnung wird vom aktuellen System nicht unterstützt. In der Applikation kann die einhändige Eingabe zweier Nutzer somit nicht von der beidhändigen eines Nutzers unterschieden werden. Fehleingaben durch gleichzeitiges Interagieren der Nutzer können folglich nicht vollständig vermieden werden. Nach diesem Zusammenhang wird Anforderung \ref{req:mehrbenutzer} maßgeblich berücksichtigt und unterstützt, jedoch nicht vollständig erreicht.
\\\\
Clipping von Modellbereichen kann Konflikte der Tiefenwahrnehmung durch Stereoparallaxe vermeiden \cite{ardouin:2011}. Das in Kapitel \ref{chp:freischneiden} beschriebene Konzept zur Beschneidung der virtuellen Geometrie, durch einen perspektivabhängigen Ausschnittzylinder nutzt diese Erkenntnis. Abschnitt \ref{sec:vorteile_und_limitierungen_freischneiden} beschreibt, dass die Wahl und Orientierung des Volumens für diesen Prozess nicht der realen Form des menschlichen Arms entspricht. Hierdurch kann nicht vollständig sichergestellt werden, dass der Anwender negativ-parallaxe Modellbereiche nicht berührt. Entstehende Wahrnehmungskonflikte wurden nach Anforderung \ref{req:wahrnehmungskonflikte} abgeschwächt, konnten jedoch nicht gänzlich vermieden werden.
\\\\
Die in die Implementierung der Touch Navigation einbezogenen Techniken ermöglichen dem Nutzer flexiblen Umgang mit virtuellen Inhalten. Hierbei wird die Auswirkungen tiefen-anpassender Gesten durch Distanz zwischen Bildschirm und Geometrie skaliert. Angewendete Transformationen sind sowohl im Nahfeld-Bereich als auch auf große Distanz präzise Möglich, was das System im Kontext des 3D Pitoti Projekts effektiv nutzbar macht. Anforderung \ref{req:interaktionsziele} wurde nach diesen Kriterien erreicht.
\\\\
Eingebundene Manipulationsformen orientieren größtenteils am von Hancock et al. vorgestellten Konzept der orthogonalen Verbindung zwischen Geometrie und Bildschirm \cite{hancock:2007,hancock:2009}. Dies unterstützt das Gefühl Modelle direkt zu Berühren. Nutzer können natürliche Bewegungsabfolgen demnach leicht auf angebotene Navigationsstrategien anwenden. In Kapitel \ref{chp:wechsel_zwischen_interaktionstechniken} wird der Wechsel zwischen den Navigationsmodi vorgestellt. Während die einzelnen Gesten auch für unerfahrene Nutzer leicht steuerbar scheinen, ist das Einprägen der Routine zur Auswahl einer Navigationstechnik nach den in Abschnitt \ref{chp:wechsel_zwischen_interaktionstechniken} Zusammenhängen fraglich. Anforderung \ref{req:intuitiv_benutzbar} wurde im Arbeitsprozess maßgeblich berücksichtigt. Es sollte eine Studie zur Bewertung der Touch Navigation hinsichtlich dieser Anforderung durchgeführt werden. 
\\\\
Die entwickelten Navigationstechniken ermöglichen die Manipulation aller sechs Freiheitsgrade der Transformation im dreidimensionalen Raum. Durch RST+L bietet die gleichzeitige Kontrolle über x- und y-Translation, z-Rotation und uniforme Skalierung, erweitert durch die implizite Tiefenebnung der \emph{Levelling} Strategie. Mit 3D Rotation sind die Drehungen um x-, y- und z-Achse getrennt zu bedienen. Der 3D Translations-Modus dient zur expliziten Kontrolle der x-, y- und z-Komponente bei der Bewegungstransformation. Die alleinige Skalierungsanpassung ist nach den in Kapitel \ref{chp:implizite_navigation} beschriebenen Voraussetzungen bei RST+L nur nach Erreichung der Interaktionsziels von \emph{Levelling} möglich. Jedoch kann dies durch menüseitige Filterung der Interaktionsmodi von der Anwendung bei Rotation, Skalierung und Translation im Bildraum ausgeschlossen werden. Somit ist der Skalierungsfaktor auch getrennt von anderen Transformationen konfigurierbar. Anforderung \ref{req:getrennte_bedienung_der_dof} gilt demnach als erfüllt.
\\\\
RST+L gibt dem Anwender die Möglichkeit bekannte Strategien der 2D Touch Interaktion für die Applikationssteuerung einzusetzen. Währenddessen sorgt die Errechnung von Manipulationsparametern bei der \emph{Levelling} Strategie für Ebnung von visuell relevanten Inhalten. Nach Bruder et al. ist die Analyse und Interaktion mit stereoskopischen Geometrien vor allem durch nahe der Bildebene liegende Objekte gegeben. Die Heranführung der vom Nutzer berührten Modelle an die Projektionsfläche ist folglich als Interaktionsziel zu sehen. Nach diesem Zusammenhang ist von der Erfüllung von Anforderung \ref{req:implizite_kontrolle} auszugehen.
\\\\
Zur Visualisierung der Interaktionsprozesse und der taktilen Eingabe wurden verschiedene virtuelle Darstellungen eingeführt. Diese verdeutlichen die Auswirkungen der Interaktion mit der Bildfläche. Sie  verstärken die Verständlichkeit der einzelnen Navigationstechniken und machen fehlerhafte Verarbeitungsergebnisse des MSER Hand und Finger Trackings sichtbar. Somit konnte Anforderung \ref{req:visueller_output} durch das System erreicht werden.


\section{Hypothesen}
\label{sec:hypothesen}

	\begin{hypothese}
	\label{hyp:fehleingaben}
		Fehleingaben der Nutzer können nur durch eine Verbesserung der Eingabeanalyse behoben werden.
	\end{hypothese}

Das MSER Finger und Hand Tracking ist auf die Finger-Hand Zuordnung begrenzt. Dieses Kriterium genügt nicht um Aussagen über die Zugehörigkeit von Eingabepositionen zu einem bestimmten Nutzer zu tätigen. Dadurch können Fehleingaben durch den gleichzeitigen Input mehrerer Anwender von der Applikation nicht vollständig ausgeschlossen werden. Genauer Aufschluss über die Hand-Nutzer Relation ist die Voraussetzung für die Entwicklung von Mechanismen zur Vermeidung beschriebener Mehrbenutzerkonflikte. Die Verbesserung der Eingabeanalyse könnte die Extrahierung dieser Daten ermöglichen.

	\begin{hypothese}
		\label{hyp:konflikte}
		Konflikte der Tiefenwahrnehmung können durch Volume Clipping behoben werden.
	\end{hypothese}
	
Auch wenn die für die See-Through Technik verwendete Ausschnittgeometrie nicht dem menschlichen Arm entspricht, hat sich die Strategie dennoch als große Unterstützung für die kollisionsfreie Tiefenwahrnehmung gezeigt. Eine präzisere Abbildung physischer Gegebenheiten über dem Tisch könnte zu einer geeigneten Konstruktion des Clipping Volumens beitragen.

	\begin{hypothese}
		\label{hyp:unerfahrene}
		Das entwickelte System ist auch für unerfahrene Nutzer leicht zu bedienen.
	\end{hypothese}
	
Abschnitt \ref{sec:diskussion_mser} spricht einige Probleme an, welche durch die Touch Erkennung entstehen. Die Auswirkung dieser beeinflusst drastisch den Umgang mit der Touch Navigation. Aus diesem Grund wurde im Rahmen dieser Arbeit keine Nutzerstudie durchgeführt. Es ist von einer Verzerrung  der Ergebnisse einer potentiellen Studie auszugehen. Testpersonen hätten die offensichtlichen Fehlfunktionen bei der Eingabeauswertung möglicherweise in die Bewertung des Navigationssystems eingebracht. Dieser Zusammenhang stellt die Nützlichkeit der Studienergebnisse zur Evaluierung der Touch Gesten in Frage. Für valide Ergebnisse sollte demzufolge eine Verbesserung der Input Verarbeitung erfolgen.

	\begin{hypothese}
		\label{hyp:levelling}
		Der Umgang mit \emph{Levelling} ist leicht erlernbar und effektiv bei der 3D Multi-Touch Interkation.
	\end{hypothese}
	
Die Berechnung des \emph{Rotation-Levelling} ist gegenüber dem Mapping expliziter Navigationstechniken komplex. Durch den visuellen Effekt der Manipulation wird diese Komplexität jedoch vor dem Nutzer verborgen. Der Zusammenhang die Geometrie durch \emph{Levelling} auf der Bildfläche zu ebnen erscheint schlüssig und leicht zu verinnerlichen. Vor allem im Umgang mit komplexen Oberflächenstrukturen zeigt sich eine automatische Ausrichtungskorrektur anhand nutzerdefinierter Punkte als effektiv.

%------------------------------------------------

\chapter{Fazit}
\label{chp:fazit}
Hier wird das Kapitel Fazit beschrieben. Das Kapitel besteht aus den Abschnitten \ref{sec:beitraege} und \ref{sec:weiterentwicklung}.


\section{Beiträge dieser Arbeit}
\label{sec:beitraege}

Hier steht Inhalt zu Beiträgen dieser Arbeit.


\section{Vorschläge zur Weiterentwicklung}
\label{sec:weiterentwicklung}

Hier steht Inhalt zu Vorschlägen zur Weiterentwicklung.


%------------------------------------------------
%	APPENDIX
%------------------------------------------------

\appendix

%------------------------------------------------

\chapter{Anhang 1}

Im Anhang kann auf Implementierungsaspekte wie Datenbankschemata

oder Programmcode eingegangen werden.


%------------------------------------------------
%	IMAGES APPENDIX
%------------------------------------------------

\listoffigures


%------------------------------------------------
%	TABLES APPENDIX
%------------------------------------------------

\listoftables


%------------------------------------------------
%	REFERENCES
%------------------------------------------------

\bibliographystyle{apalike}    

\bibliography{main_bib}         % verwendet main_bib.bib


%------------------------------------------------
%	END OF DOCUMENT
%------------------------------------------------

\end{document}