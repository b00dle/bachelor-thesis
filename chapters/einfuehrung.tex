\section{Mehrbenutzer VR-Systeme}
\label{sec:mehrbenutzer_vr_systeme}

Aktuelle Entwicklungen am Lehrstuhl für virtuelle Systeme ermöglichen die Darstellung von komplexen 3D Scans. Im Rahmen des EU-Forschungsprojekts 3D Pitoti werden beispielsweise umfangreiche Datensätze in verschiedenen Auflösungsstufen, zur Betrachtung von prähistorischen Felsgravuren in Val Camonica zur interaktiven Visualisierung, angeboten. 
\\\\
Um hierbei Oberflächenstruktur im Submillimeterbereich, sowie deren nähere Umgebung, kollaborativ analysieren zu können, werden verschiedene Mehrbenutzer Projektionssysteme genutzt. Eine Projektionswand wird verwendet um Inhalte im 1:1 Maßstab anzuzeigen, während ein Bildschirmtisch gleichzeitig den Überblick über das gesamte Tal liefern kann. Beide Systeme begünstigen die Betrachtung stark vergrößerter Oberflächendetails.
\\\\
Die Verteilung einer Applikation über mehrere Bildschirme wird nach Seyed et al. \cite{seyed:2013} auch als \emph{Multi-Display-Umgebung} bezeichnet. In dieser Umgebung können Nutzer in Gruppen variabler Größe, oder allein, mit den Angebotenen Darstellungen interagieren. Des Weiteren können Anwender frei zwischen verschiedenen Bildschirmen wechseln.


\section{Multi-Touch Interaktion mit dreidimensionalen Szenen}
\label{sec:multi_touch_interaktion_mit_3d_szenen}

Von einer breiten Masse der Bevölkerung werden Geräte wie Tablets, Smartphones, oder sonstige Touchscreens täglich für unterschiedliche Aufgaben genutzt. Hierbei haben sich verschiedene Gesten zur Kontrolle von zweidimensionalen Inhalten bereits etabliert. Durch diese Erfahrungen erwarten Nutzer intuitiv Eingaben an der Oberfläche eines Bildschirmtisches durchführen zu können.  Auch stereoskopische Bildinhalte sind in der Öffentlichkeit, beispielsweise durch 3D Kinofilme, längst etabliert. Für die Verbindung dieser beiden Techniken gibt es bislang jedoch keine vollständig akzeptierte Lösung.
\\\\ 
Stereoskopisches Rendering führt zu positiver, negativer, oder null Parallaxe bei der Darstellung von Bildinhalten. Das heißt virtuelle Modelle erscheinen für die Wahrnehmung unterhalb (Positiv-Parallaxe), oberhalb (Negativ-Parallaxe), oder auf (Null-Parallaxe) der Tischoberfläche. Bekannte 2D Touch Interaktionen basieren auf direktem Kontakt mit den Objekten der virtuellen Darstellung. Es entsteht eine Bindung die das Gefühl erzeugt Objekte festzuhalten. Dieser Effekt kann auf direkte Weise nur bei null Parallaxe auf eine stereoskopische Anwendung übertragen werden \cite{bruder:2013}. Außerdem ruft das Greifen der Hand in negativ parallaxe Bildarealen Störungen der Tiefenwahrnehmung hervor \cite{delariviere:2010}.
\\\\
Des Weiteren hat die Integration einer 3D Szene Auswirkungen auf die Freiheitsgrade der Manipulation. Hierbei wird die 2D Translation um die Verschiebung entlang der z-Achse erweitert. Rotation erfolgt im 3D Raum unter Berücksichtigung dreier Achsen. Im 2D Raum wird hierfür lediglich eine Achse benötigt. Da die Eingabeparameter eines Berührungspunktes der Bildschirmfläche zweidimensional sind, ergibt sich eine Herausforderung für die Entwicklung einer geeigneten 3D Manipulationsschnittstelle \cite{martinet:2012}.


\section{Ziel der Arbeit}
\label{sec:ziel_der_arbeit}

Die Verwendung von Projektionstischen zur Darstellung virtueller Bildinhalte begünstigt die Einbindung von taktilen Eingaben. Durch blickabhängige Renderingverfahren werden kollaborative Anwendungen leicht realisierbar. Ziel dieser Arbeit ist es, eine Mehrbenutzer-Schnittstelle zur touch-basierten Applikationsnavigation mit dreidimensionalen Inhalten zu entwickeln. Hierbei soll es mehreren Nutzern möglich sein gemeinsam mit den dargebotenen virtuellen Welten umzugehen. Im Kontext des 3D Pitoti EU Forschungsprojekts soll das entwickelte System die Exploration skalierbarer Oberflächenscans effektiv ermöglichen.
\\\\
Vor diesem Hintergrund werden diverse Anforderungen definiert, welche in Kapitel \ref{chp:anforderungsanalyse} näher beschrieben werden. Die zugrunde liegende Finger und Hand Erkennung basiert auf einer im Rahmen einer Masterabschlussarbeit entstandenen Verfahrenstechnik \cite{ewerling:2012}. Kapitel \ref{chp:mser} gibt Einblicke in den hierfür verwendeten \emph{Maximally Stable Extremal Regions} Algorithmus nach Matas et al. \cite{matas:2004}. Außerdem werden hardwareseitige Voraussetzungen, sowie die Anbindung an das für die Multi-Touch Navigation verwendete Framework Avango geschildert \cite{avango:2011}. Zur Implementierung geeigneter Touch Techniken gilt es verwandte Arbeiten in diesem Bereich zu analysieren und zu evaluieren. Die Auseinandersetzung mit diesem Thema wird in Kapitel \ref{chp:interaktion_mit_multitoucheingaben} geschildert. Außerdem lassen sich in der Literatur diverse Hinweise auf durch Interaktion mit stereoskopischen Projektionen entstehende Wahrnehmungskonflikte finden. In Kapitel \ref{chp:wahrnehmungskonflikte_und_loesungsansaetze} werden diese näher beleuchtet und bestehende Lösungsvorschläge diskutiert. Durch Manipulation des virtuellen Viewing Setups können Navigationsaufgaben in der Applikation realisiert werden. Hierzu sind Transformationen im dreidimensionalen Raum mit verschiedenen Freiheitsgraden zu bestimmen und anzuwenden. Kapitel \ref{chp:explizite_interaktion} stellt die im Rahmen dieser Arbeit entstandenen Multi-Touch Gesten vor, welche dem Nutzer durch verschiedene Eingabeabläufe die Parametrisierung aller Freiheitsgrade explizit ermöglichen. Implizite Navigationsansätze erweitern diese Strategien durch zusätzliche Manipulation der Applikation. Diese werden aus den ausgearbeiteten Interaktionszielen des Nutzers abgeleitet und können indirekt durch den Anwender gesteuert werden. Eine Konkretisierung dieses Konzepts durch die Einführung der \emph{Levelling}-Technik erfolgt in Kapitel \ref{chp:implizite_navigation}. Um die zur Navigation verwendeten Gesten zur Navigation zu Unterscheiden ist die Gliederung in verschiedene Eingabemodi erforderlich. Diese gilt es zu definieren und durch geeignete Mechanismen für den Nutzer auswählbar zu machen. Kapitel \ref{chp:wechsel_zwischen_interaktionstechniken} erklärt wie der Wechsel zwischen Interaktionstechniken bewerkstelligt wird. Wie in Kapitel \ref{chp:wahrnehmungskonflikte_und_loesungsansaetze} beschrieben wird, resultieren aus dem Eindringen physischer Objekte in Applikationsgeometrie Konflikte für die Wahrnehmung der Nutzer. Durch Kapitel \ref{chp:freischneiden} wird die \emph{See-Through} Technik zur Abschwächung dieser Probleme bei Touch Eingaben eingeführt und diskutiert. In Kapitel \ref{chp:applikationsstruktur} wird die Implementierung der erarbeiteten Konzepte in das Applikationssystem beschrieben. Im Anschluss daran erfolgt in Kapitel \ref{chp:diskussion} eine Diskussion zur Erfüllung der durch die Anforderungsanalyse definierten Vorgaben. Des Weiteren werden, entlang der im Arbeitsprozess aufgetretenen Beobachtungen, Hypothesen über das entwickelte System abgeleitet. Abschließend fasst Kapitel \ref{chp:fazit} die Beiträge dieser Bachelorarbeit zusammen und gibt einen Ausblick auf mögliche zukünftige Arbeiten.