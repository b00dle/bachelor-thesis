\section{Mehrbenutzer VR-Systeme}
\label{sec:mehrbenutzer_vr_systeme}

Aktuelle Entwicklungen am Lehrstuhl für virtuelle Systeme ermöglichen die Darstellung von komplexen 3D Scans. Im Rahmen des EU-Forschungsprojekts 3D Pitoti werden beispielsweise umfangreiche Datensätze in verschiedenen Auflösungsstufen, zur Betrachtung von prähistorischen Felsgravuren in Val Camonica zur interaktiven Visualisierung, angeboten. 
\\\\
Um hierbei Oberflächenstruktur im Submillimeterbereich, sowie deren nähere Umgebung, kollaborativ analysieren zu können, werden verschiedene Mehrbenutzer Projektionssysteme genutzt. Eine Projektionswand wird verwendet um Inhalte im 1:1 Maßstab anzuzeigen, während ein Bildschirmtisch gleichzeitig den Überblick über das gesamte Tal liefern kann. Beide Systeme begünstigen die Betrachtung stark vergrößerter Oberflächendetails.
\\\\
Die Verteilung einer Applikation über mehrere Bildschirme wird nach Seyed et al. \cite{seyed:2013} auch als \emph{Multi-Display-Umgebung} bezeichnet. In dieser Umgebung können Nutzer in Gruppen variabler Größe, oder allein, mit den Angebotenen Darstellungen interagieren. Des Weiteren können Anwender frei zwischen verschiedenen Bildschirmen wechseln.


\section{Multi-Touch Interaktion mit dreidimensionalen Szenen}
\label{sec:multi_touch_interaktion_mit_3d_szenen}

Von einer breiten Masse der Bevölkerung werden Geräte wie Tablets, Smartphones, oder sonstige Touchscreens täglich für unterschiedliche Aufgaben genutzt. Hierbei haben sich verschiedene Gesten zur Kontrolle von zweidimensionalen Inhalten bereits etabliert. Durch diese Erfahrungen erwarten Nutzer intuitiv Eingaben an der Oberfläche eines Bildschirmtisches durchführen zu können.  Auch stereoskopische Bildinhalte sind in der Öffentlichkeit, beispielsweise durch 3D Kinofilme, längst etabliert. Für die Verbindung dieser beiden Techniken gibt es bislang jedoch keine vollständig akzeptierte Lösung.
\\\\ 
Stereoskopisches Rendering führt zu positiver, negativer, oder null Parallaxe bei der Darstellung von Bildinhalten. Das heißt virtuelle Modelle erscheinen für die Wahrnehmung unterhalb (positive Parallaxe), oberhalb (negative Parallaxe), oder auf (null Parallaxe) der Tischoberfläche. Bekannte 2D Touch Interaktionen basieren auf direktem Kontakt mit den Objekten der virtuellen Darstellung. Es entsteht eine Bindung die das Gefühl erzeugt Objekte festzuhalten. Dieser Effekt kann auf direkte Weise nur bei null Parallaxe auf eine stereoskopische Anwendung übertragen werden (\cite{bruder:2013}). Außerdem ruft das Greifen der Hand in negativ parallaxe Bildarealen Störungen der Tiefenwahrnehmung hervor (\cite{delariviere:2010}).
\\\\
Des Weiteren hat die Integration einer 3D Szene Auswirkungen auf die Freiheitsgrade der Manipulation. Hierbei wird die 2D Translation um die Verschiebung entlang der z-Achse erweitert. Rotation erfolgt im 3D Raum unter Berücksichtigung dreier Achsen. Im 2D Raum wird hierfür lediglich eine Achse benötigt. Da die Eingabeparameter eines Berührungspunktes der Bildschirmfläche zweidimensional sind, ergibt sich eine Herausforderung für die Entwicklung einer geeigneten 3D Manipulationsschnittstelle (\cite{martinet2012}).


\section{Ziel der Arbeit}
\label{sec:ziel_der_arbeit}

Die Verwendung von Projektionstischen zur Darstellung stereoskopischer Bildinhalte begünstigt die Verwendung von taktilen Eingaben. Durch blickabhängige Renderingverfahren werden kollaborative Anwendungen leicht realisierbar. Ziel dieser Arbeit ist es, eine Mehrbenutzer-Schnittstelle zur touch-basierten Applikationsinteraktion mit dreidimensionalen Inhalten zu entwickeln. 