Hier wird das Kapitel Anforderungsanalyse beschrieben. Das Kapitel besteht aus den Abschnitten \ref{sec:mehrbenutzer_touch_eingaben}, \ref{sec:tiefenwahrnehmung}, \ref{sec:interaktionsziele_und_freiheitsgrade}, \ref{sec:kontrolle_der_freiheitsgrade} und \ref{sec:visuelle_rueckmeldung}.


\section{Mehrbenutzer Touch-Eingaben}
\label{sec:mehrbenutzer_touch_eingaben}

Hier steht Inhalt zu Mehrbenutzer Touch-Eingaben.


\section{Tiefenwahrnehmung durch Stereo- und Bewegungsparalaxe}
\label{sec:tiefenwahrnehmung}

Hier steht Inhalt zu Tiefenwahrnehmung durch Stereo- und Bewegungsparalaxe. 


\section{Interaktionsziele und involvierte Freiheitsgrade}
\label{sec:interaktionsziele_und_freiheitsgrade}

Hier steht Inhalt zu Interaktionszielen und involvierten Freiheitsgraden. 


\section{Explizite und implizite Kontrolle der involvierten Freiheitsgrade}
\label{sec:kontrolle_der_freiheitsgrade}

Hier steht Inhalt zu Expliziter und impliziter Kontrolle der involvierten Freiheitsgrade.


\section{Visuelle Rückmeldung}
\label{sec:visuelle_rueckmeldung}

Hier steht Inhalt zu Visueller Rückmeldung. 