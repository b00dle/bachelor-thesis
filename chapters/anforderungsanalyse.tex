Hier wird das Kapitel Anforderungsanalyse beschrieben. Das Kapitel besteht aus den Abschnitten \ref{sec:mehrbenutzer_touch_eingaben}, \ref{sec:tiefenwahrnehmung}, \ref{sec:interaktionsziele}, \ref{sec:kontrolle_der_freiheitsgrade} und \ref{sec:visuelle_rueckmeldung}.


\section{Mehrbenutzer Touch-Eingaben}
\label{sec:mehrbenutzer_touch_eingaben}

Anforderung: Fehleingaben durch gleichzeitiges Interagieren verschiedener Nutzer mit dem Bildschirmtisch sollten vermieden werden.
\\\\
Die meisten Interaktionswerkzeuge können von nicht mehr als einem Nutzer bedient werden. Selbst wenn die Auswirkung der Manipulation die ganze Gruppe betrifft, so obliegt der Umgang mit dem Werkzeug einer Person. Im Gegensatz dazu, steht der Projektionstisch allen Nutzern gleichermaßen zur Verfügung. Da eine Vielzahl verschiedener Fingerpositionen auf dem Touchtisch registriert werden kann, besteht folglich die Möglichkeit, dass der Tisch zur gleichen Zeit von mehreren Personen Eingaben erhält. Diese Eingabekombinationen können leicht zu Manipulationen der Applikation führen, welche vom Nutzer nicht beabsichtigt waren und somit vermieden werden sollten.
\\\\
Abbildung XX zeigt ein konkretes Anwendungsszenario. Hier versammeln sich eine bestimmte Anzahl Nutzer um gemeinsam Eigenschaften einer virtuellen Welt zu untersuchen. Während ein Nutzer beidhändig eine Interaktionsgeste ausführt, setzt ein anderer Nutzer seine Hand auf einen markanten Inhalt der Szene. Dabei berührt er ebenfalls die Tischfläche. In diesem Fall sollte sichergestellt sein, dass das System die Semantik der Touchpunkte erkennt und eine sinnvolle Entscheidung bezogen auf die abgeleitete Interaktion trifft. 


\section{Tiefenwahrnehmung durch Stereo- und Bewegungsparalaxe}
\label{sec:tiefenwahrnehmung}

Anforderung: Konflikten der Tiefenwahrnehmung durch Berührung negativ paralaxer Bildareale sollte entgegen gewirkt werden.
\\\\
Anforderung: Die Interaktion mit positiv und negativ paralaxen Objekten sollte nicht durch Bewegungsparalaxe erschwert werden. 


\section{Interaktionsziele}
\label{sec:interaktionsziele}

Anforderung: Manipulationen der Applikation, sollten basierend auf dem Interaktionsziel des Nutzers bestimmt werden.


\section{Explizite und implizite Kontrolle der involvierten Freiheitsgrade}
\label{sec:kontrolle_der_freiheitsgrade}

Anforderung: Das System sollte auch für unerfahrene Nutzer leicht benutzbar sein.
\\\\
Anforderung: Es sollte dem Nutzer möglich sein, alle für das jeweilige Interaktionsziel erforderlichen Freiheitsgrade getrennt voneinander zu bedienen.
\\\\
Anforderung: implizite kontrolle????



\section{Visuelle Rückmeldung}
\label{sec:visuelle_rueckmeldung}

Anforderung: Der Umgang mit der Schnittstelle sollte mit visuellem Output unterstützt werden